% !TEX TS-program = xelatex

\documentclass{book}
\usepackage{xepersian} % Package for Persian (Farsi) support in XeLaTeX
\settextfont{Times New Roman} % Set font

\begin{document}

\begin{center}
\textbf{لینوکس از صفر} \\ 
نسخه ۳.۱۲ \\ 
منتشر شده در ۱۳ فوریه ۲۰۲۵ \\ 
ایجاد شده توسط Gerard Beekmans \\ 
ویرایشگر ارشد: Dubbs Bruce \\ 
\end{center}

\newpage

\begin{center}
حق نشر © ۱۹۹۹-۲۰۲۵ Gerard Beekmans \\ 
تمامی حقوق محفوظ است. \\ 
این کتاب تحت مجوز Commons Creative منتشر شده است. \\ 
دستورالعمل‌های رایانه‌ای را می‌توان تحت مجوز MIT از کتاب استخراج کرد. \\ 
لینوکس® یک علامت تجاری ثبت‌شده‌ی Trovalds Linus است.
\end{center}

\chapter{مقدمه}
\newpage

\section{پیشگفتار}

مسیر من برای یادگیری و درک بهتر لینوکس در سال ۱۹۹۸ آغاز شد. دز أن زمان، اولین توزیع لینوکس را نصب کردم و به سرعت کنجکاو مفهوم کلی و فلسفه پشت لینوکس شدم.
\newline
\newline
همیشه راه‌های زیادی برای انجام یک تسک وجود دارد و این قضیه برای لینوکس نیز صادق است. تعداد زیادی در طول سال‌ها وجود داشته‌اند. برخی هنوز هم وجود دارند، برخی تغییر ماهیت داده اند و برخی به تاریخ پیوسته اند. ولی همه آن‌ها برای رفع نیاز‌های مخاطب خود به صورت متفاوت طراحی شده اند. به دلیل این که راه‌های زیادی برای به ثمر رسیدن یک هدف وجود دارد، به این نتیجه رسیدم که نباید خود را محدود به یک توزیع کنم. ما قبل از کشف لینوکس مشکلاتی با مابقی سیستم‌های عامل داشتیم و چاره‌ای جز استفاده از آن‌ها نداشتیم، چه راضی بودیم چه ناراضی. با به روی کار آمدن لینوکس، مفهوم انتخاب ظهور کرد. اگر از چیزی خوشتان نمی‌آمد، آزاد بودید، یا حتی تشویق می‌شدید که آن را تغییر دهید.
\newline
\newline
من تعداد زیادی از توزیع‌های لینوکس را امتحان کردم و نمی‌توانستم یکی را بین آن‌ها انتخاب کنم. هر کدام از آن‌ها ویژگی‌های منحصر به فرد خود را داشتند و این به معنی خوب یا بد بودن آن‌ها نیست، بلکه انتخاب یک توزیع لینوکس کاملا یک انتخاب سلیقه‌ای است. با وجود تعداد زیادی از توزیع‌ها، به این نتیجه رسیدم که یک توزیع خاص برای رفع تمام نیاز‌های من وجود ندارد. بنابراین تصمیم گرفتم که توزیع لینوکس خودم را پایه‌ریزی که براساس نیاز‌های من ساخته شده باشد.
\newline
\newline
برای تحقق این کار باید همه چیز را خودم از صفر (سورس کد) کامپایل می‌کردم و از پکیج‌های باینری از پیش کامپایل شده استفاده نمی‌کردم. این توزیع “بی نقص” لینوکس می‌بایست شامل نقاط قوت بسیاری از سیستم‌ها می‌بود و در عوض نقاط ضعف را پوشش نمی‌داد. در ابتدا، این ایده به نسبتا دلهره آور بود ولی من به این ایده که این سیستم می‌تواند ساخته شود متعهد ماندم.
\newline
\newline
پس از پشت سر گذاشتن مسائلی مانند وابستگی‌های درهم تنیده و خطا‌های زمان اجرا، بالاخره موفق به ساخت یک سیستم لینوکس شخصی‌سازی شده، شدم. در آن زمان، این توزیع، مانند هر توزیع دیگر موجود، کاملا قابل استفاده بود، ولی این سیستم، ساخته خودم بود. این موضوع که توانستم یک همچین سیستمی را پیاده سازی کنم، بسیار رضایت‌بخش بود و بهترین بخش آن، ساخت هر قطعه از نرم‌افزار توسط خودم بود.
\newline
\newline
پس از به اشتراک گذاشتن اهداف و تجربه‌هایم با مابقی اعضای جامعه لینوکس، متوجه شدم که علاقه شدیدی به این ایده‌ها وجود دارد. به سرعت مشخص شد که این چنین سیستم‌های سفارشی بر پایه لینوکس نه تنها برای برآوردن نیاز‌های خاص کابران، بلکه به عنوان یک سیستم ایده‌آل برای برنامه‌نویسان و مدیران سیستم برای ارتقای مهارت‌های لینوکس (موجود) خود عمل می‌کنند. با توجه به این علاقه، پروژه لینوکس از صفر متولد شد.
\newline
\newline
کتاب “لینوکس از صفر” هسته مرکزی این پروژه است. این کتاب شامل پیش‌زمینه‌ها و دستورالعمل‌های لازم برای طراحی و ساخت سیستم خود می‌شود. با وجود این که این کتاب یک قالب برای ساخت یک سیستم کاربردی را فراهم می‌کنم، شما آزاد هستید که دستورالعمل‌ها را مطابق با میل خود تغییر دهید و این بخش مهمی از این پروژه است. شما تحت کنترل می‌مانید، ما فقط به شما کمک می‌کنیم که مسیر خود را آغاز کنید.
\newline
\newline
من عمیقا امیدوارم که شما زمان خوبی را در حین ساخت سیستم لینوکس از صفر خودتان سپری کنید و از مزایای بی‌شمار داشتن 
سیستمی که واقعا متعلق به خودتان است، لذت ببرید.
\newline
\newline
\begin{flushleft}
Gerard Beekmans \\
gerard@linuxfromscratch.org
\end{flushleft}

\end{document}
